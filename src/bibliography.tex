\section*{БИБЛИОГРАФИЧЕСКИЙ СПИСОК}
\addcontentsline{toc}{section}{Библиографический список}

\subsection*{Список использованных источников}
\addcontentsline{toc}{subsection}{Список использованных источников}

\begingroup

\renewcommand{\section}[2]{}

\begin{thebibliography}{}

\bibitem{moluch}
Сагиндыкова А. С., Тугамбекова М. А.
\emph{Актуальность дистанционного образования // Молодой ученый. — 2015. — №20. — С. 495-498.
[Электронный ресурс]. – Режим доступа:
\href{https://moluch.ru/archive/100/20703/}{https://moluch.ru/archive/100/20703/}
(дата обращения: 05.01.2019).}

\bibitem{the-steppe}
\emph{10 онлайн-платформ, которые прокачают ваш мозг.
[Электронный ресурс]. – Режим доступа:
\href{https://the-steppe.com/news/gorod/2018-05-18/10-onlayn-platform-kotorye-prokachayut-vash-mozg}{https://the-steppe.com/news/gorod/2018-05-18/10-onlayn-platform-kotorye-prokachayut-vash-mozg}
(дата обращения: 05.01.2019).}

\bibitem{udacity}
\emph{Udacity - Википедия. [Электронный ресурс]. – Режим доступа:
\href{https://ru.wikipedia.org/wiki/Udacity}{https://ru.wikipedia.org/wiki/Udacity}
(дата обращения: 05.01.2019).}

\bibitem{docker}
\emph{Docker - Википедия. [Электронный ресурс]. – Режим доступа:
\href{https://ru.wikipedia.org/wiki/Docker}{https://ru.wikipedia.org/wiki/Docker}
(дата обращения: 09.01.2019).}

\bibitem{angular}
\emph{10 преимуществ использования фреймворка Angular.js при разработке веб-приложений [Электронный ресурс]. – Режим доступа:
\href{https://stfalcon.com/ru/blog/post/why-use-angularjs-for-webapps}{https://stfalcon.com/ru/blog/post/why-use-angularjs-for-webapps}
(дата обращения: 09.01.2019).}







\bibitem{}
Neil Smyth,
\emph PHP Essentials,
Payload Media,
2010.

\bibitem{}
  Andi Gutmans, Stig Sæther Bakken, and Derick Rethans,
  \emph PHP 5 Power Programming,
  Prentice Hall PTR,
  2004.

\bibitem{}
  David Sklar; Adam Trachtenberg,
  \emph PHP Cookbook: Solutions and Examples for PHP Programmers,
  O'Reilly Media; 2 edition (August 1, 2006), eBook (2008).

\bibitem{}
  Samisa Abeysinghe,
  \emph RESTful PHP Web Services,
  Packt Publishing (October 30, 2008).

\bibitem{}
  Paul Hudson,
  \emph Practical PHP Programming,
  tuxradar.com (2009).

\bibitem{}
  Mario Lurig,
  \emph PHP Reference: Beginner to Intermediate PHP5,
  Lulu.com; 1st edition (April 11, 2008).

\bibitem{}
  Baron Schwartz, Peter Zaitsev, Vadim Tkachenko, Jeremy Zawodny, Arjen Lentz, Derek J. Balling,
  \emph High Performance MySQL: Optimization, Backups, Replication, and More,
  O'Reilly Media (2008).


\bibitem{}
  Paul DuBois,
  \emph MySQL Cookbook,
  O'Reilly Media (2006)

\bibitem{}
  Charles Bell, Mats Kindahl, Lars Thalmann,
  \emph MySQL High Availability: Tools for Building Robust Data Centers,
  O'Reilly Media (2010)

\bibitem{}
  Michael Kruckenberg, Jay Pipes,
  \emph Pro MySQL (The Expert's Voice in Open Source),
  Apress Media (2005)

\bibitem{}
  Seyed Tahaghoghi, Hugh Williams,
  \emph Learning MySQL,
  O'Reilly Media (2006)

\bibitem{mihnuk07}
  Михнюк Т.Ф.,
  \emph{Охрана труда и основы экологии}.
  Минск: ``Вышэйшая школа'',
  2007.

\bibitem{devisilov09}
  Девисилов В.А.,
  \emph{Охрана труда}.
  М.: ФОРУМ,
  2009.

\bibitem{belov09}
  Белов С.В.,
  \emph{Безопасность жизнедеятельности}.
  М.: Высшая школа,
  2007.

\bibitem{decree432}
  Указ Президента РБ №432 от 31 августа 2009 года,
  \emph{О некоторых вопросах приобретения имущественных прав на результаты научно-технической деятельности и распоряжения этими правами}.
  \href{http://president.gov.by/press76885.html}{http://president.gov.by/press76885.html}.

\bibitem{palitsyn06}
  Палицын В.А.,
  \emph{Технико-экономическое обоснование дипломных проектов. В 4-х частях. Часть 4: проекты программного обеспечения}.
  Мн.: БГУИР,
  2006.

\bibitem{palitsyn06}
  Светлицкий, И.С.,
  \emph{Экономическая теория: Электронный учебно-методический комплекс для студентов всех неэкономических специальностей}.
  Мн.: БГУИР,
  2006.

\end{thebibliography}

\subsection*{Список публикаций соискателя}
\addcontentsline{toc}{subsection}{Список публикаций соискателя}

\begin{thebibliography}{}

\bibitem{}
  Теслюк В.Н., Танасюк О.О.
  \emph{Разработка образовательной платформы для хранения, обработки, просмотра видеоматериалов
  // Студенческий форум: электрон. научн. журн. 2019. № 1(52).
  [Электронный ресурс]. – Режим доступа:
  \href{https://nauchforum.ru/journal/stud/52/45530}{https://nauchforum.ru/journal/stud/52/45530}
  (дата обращения: 05.01.2019).}

\end{thebibliography}{}

\endgroup