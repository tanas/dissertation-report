\documentclass[pdftex,a4paper,14pt,english,russian]{extarticle}

\usepackage[top=2cm,bottom=20mm,left=3cm,right=10mm]{geometry}
\usepackage[T2A]{fontenc}
\usepackage[utf8]{inputenc}
\usepackage[english,russian]{babel}
\usepackage[pdftex]{hyperref}
\usepackage{indentfirst}
\usepackage[pdftex]{graphicx}
\usepackage{amsmath}
\usepackage{amssymb}
\usepackage{ragged2e}
\usepackage{multirow}
\usepackage{algorithmic}
\usepackage[boxed]{algorithm}
\usepackage{tabularx}
\usepackage{listings}
\usepackage{xtab}
\usepackage{subfig}
\usepackage{hyphenat}
\usepackage{setspace}
\usepackage{fancyhdr}
\usepackage{longtable}
\usepackage{float}
\usepackage{placeins}
\usepackage{flafter}
\usepackage{pscyr}
\usepackage{enumitem}
\usepackage{titlesec}
\usepackage{etoolbox}

% Уменьшение интервала между секциями в оглавлении
\makeatletter
\patchcmd{\l@section}% <cmd>
{\addvspace{1.0em \@plus\p@}}% <search>
{}% <replace>
{}{}% <success><failure>
\makeatother

\titleformat{\section}[block]{\centering\fontfamily{ftm}\fontsize{19}{15}\selectfont\bfseries}{\thesection}{1ex}{}
\titleformat{\subsection}[block]{\hspace{4ex}\fontfamily{ftm}\fontsize{17}{15}\selectfont\bfseries}{\thesubsection}{1ex}{}
\titleformat{\subsubsection}[block]{\hspace{4ex}\fontfamily{ftm}\fontsize{15}{15}\selectfont\bfseries}{\thesubsubsection}{1ex}{}

\setlength\parindent{4ex}

\pagestyle{fancyplain}
\fancyhf{}
\renewcommand{\headrulewidth}{0pt}
\cfoot{\fancyplain{}{\thepage}}
\linespread{1.2}
\floatname{algorithm}{Алгоритм}

\renewcommand{\rmdefault}{ftm} % Times New Roman

% Зачем: Задание подписей, разделителя и нумерации частей рисунков
% Почему: Пункт 2.5.5 Требований по оформлению пояснительной записки.
\DeclareCaptionLabelFormat{stbfigure}{Рисунок #2}
\DeclareCaptionLabelFormat{stbtable}{Таблица #2}
\DeclareCaptionLabelFormat{stblstlisting}{Листинг #2}
\DeclareCaptionLabelSeparator{stb}{~--~}
\captionsetup{labelsep=stb}
\captionsetup[figure]{labelformat=stbfigure,justification=centering}
\captionsetup[lstlisting]{labelformat=stblstlisting,justification=raggedleft}
\captionsetup[table]{labelformat=stbtable,justification=raggedright}
\renewcommand{\thesubfigure}{\asbuk{subfigure}}

\addto\captionsrussian{\renewcommand{\contentsname}{Оглавление}}