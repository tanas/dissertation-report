\documentclass[pdftex,a4paper,14pt,english,russian]{extarticle}

\usepackage[top=2cm,bottom=27mm,left=3cm,right=18mm]{geometry}
\usepackage[T2A]{fontenc}
\usepackage[utf8]{inputenc}
\usepackage[english,russian]{babel}
\usepackage[pdftex]{hyperref}
\usepackage{indentfirst}
\usepackage[pdftex]{graphicx}
\usepackage{amsmath}
\usepackage{amssymb}
\usepackage{ragged2e}
\usepackage{multirow}
\usepackage{algorithmic}
\usepackage[boxed]{algorithm}
\usepackage{tabularx}
\usepackage{listings}
\usepackage{xtab}
\usepackage{subfig}
\usepackage{hyphenat}
\usepackage{setspace}
\usepackage{fancyhdr}
\usepackage{longtable}
\usepackage{float}
\usepackage{placeins}
\usepackage{flafter}
\usepackage{pscyr}

\pagestyle{fancyplain}
\fancyhf{}
\renewcommand{\headrulewidth}{0pt}
\rfoot{\fancyplain{}{\thepage}}
\linespread{1.2}
\floatname{algorithm}{Алгоритм}

% Зачем: Выбор шрифта по-умолчанию. 
% Почему: Пункт 2.1.1 Требований по оформлению пояснительной записки.
% Примечание: В требованиях не указан, какой именно шрифт использовать. По традиции используем TNR.
\renewcommand{\rmdefault}{ftm} % Times New Roman

% Зачем: для оформления введения и заключения, они должны быть выровнены по центру.
% Почему: Пункты 1.1.15 и 1.1.11 Требований по оформлению пояснительной записки.
\makeatletter
\newcommand\sectioncentered{%
  \clearpage\@startsection {section}{1}%
    {\z@}%
    {-1em \@plus -1ex \@minus -.2ex}%
    {1em \@plus .2ex}%
    {\centering\hyphenpenalty=10000\normalfont\large\bfseries\MakeUppercase}%
    }
\makeatother

% Зачем: Не отображать номер секции в содержании
\makeatletter
\let\latexl@section\l@section
\def\l@section#1#2{\begingroup\let\numberline\@gobble\latexl@section{#1}{#2}\endgroup}
\def\@seccntformat#1{\ifcsname #1format\endcsname\else\csname the#1\endcsname\quad\fi}
\def\sectionformat{}
\makeatother


% Зачем: Задание подписей, разделителя и нумерации частей рисунков
% Почему: Пункт 2.5.5 Требований по оформлению пояснительной записки.
\DeclareCaptionLabelFormat{stbfigure}{Рисунок #2}
\DeclareCaptionLabelFormat{stbtable}{Таблица #2}
\DeclareCaptionLabelFormat{stblstlisting}{Листинг #2}
\DeclareCaptionLabelSeparator{stb}{~--~}
\captionsetup{labelsep=stb}
\captionsetup[figure]{labelformat=stbfigure,justification=centering}
\captionsetup[lstlisting]{labelformat=stblstlisting,justification=raggedleft}
\captionsetup[table]{labelformat=stbtable,justification=raggedright}
\renewcommand{\thesubfigure}{\asbuk{subfigure}}
