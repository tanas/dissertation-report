\section*{ГЛАВА 1\\ ОБЗОР ПРЕДМЕТНОЙ ОБЛАСТИ}
\setcounter{section}{1}
\addcontentsline{toc}{section}{ГЛАВА 1}
\addcontentsline{toc}{section}{ОБЗОР ПРЕДМЕТНОЙ ОБЛАСТИ}

В данном разделе будет произведён обзор предметной области задачи,
решаемой в рамках диссертационного проекта.

\subsection{Дистанционное образование}

Дистанционное обучение — это способ обучения на расстоянии, при котором преподаватель и
обучаемые физически находятся в различных местах. При таком виде обучения у людей, которые
обременены семейными и деловыми заботами и не имеют возможности посещать традиционные занятия
появляется шанс получить качественные услуги по обучению. Дистанционное обучение отвечает
требованиям современной жизни, особенно, если учесть не только транспортные расходы, но и
расходы на организацию всей системы очного обучения. Отсюда все повышающийся интерес к
дистанционному обучению, к его самым различным формам, необходимым на протяжении
всей жизни человека.\cite{moluch}

Дистанционное образование — образование, реализуемое посредством дистанционного обучения.
Характерными чертами дистанционного образования являются:
\begin{itemize}[wide,topsep=0pt]
  \itemsep0em
  \item гибкость — обучаемые в системе дистанционного образования работают в удобном месте и в удобном темпе,
  в удобное для себя время, где каждый может учиться столько, сколько ему лично необходимо для
  освоения предмета и получения необходимых экзаменов по выбранным курсам;
  \item модульность — каждый курс создает целостное представление об определенной предметной области,
  что позволяет формировать учебную программу по индивидуальным и групповым потребностям; преподаватель в дистанционном обучении — это координатор познавательной деятельности
  обучающегося и менеджер его учебного процесса;
  \item специализированный контроль качества обучения —
  используются дистанционно организованные экзамены, собеседования, практические, курсовые и
  проектные работы, компьютерные интеллектуальные тестирующие системы;
\end{itemize}

Мотивация имеет большое значение в дистанционном обучении. Именно мотивация к получению действительно прочных знаний
является движущей силой для дистанционного обучения. Дело в том, что человек, получивший диплом,
но не подтвердивший своих знаний и навыков на практике, после того как был принят на работу,
не имеет никаких шансов надеяться на то, что работодатель будет удовлетворен его деятельностью.
Скорее наоборот. Он будет уволен и его место займет тот, кто действительно получил прочные и
реальные знания.

Характерными чертами дистанционного образования являются модульность, изменение роли
преподавателя (в значительной степени связанное с разделением функций разработчиков курсов,
тьюторов и др.), использование специализированных технологий и средств обучения и т. д.

Отличия и плюсы дистанционного образования это
\begin{itemize}[wide,topsep=0pt]
  \itemsep0em
  \item постоянный контакт с преподавателем (тьютором), возможность оперативного обсуждения с ним
  возникающих вопросов, как правило, при помощи средств телекоммуникаций;
  \item возможность организации дискуссий, совместной работы над проектами и других видов групповых
  работ в ходе изучения курса и в любой момент (при этом группа может состоять как
  из компактно проживающих в одной местности студентов, так и быть распределенной).
  \item передача теоретических материалов учащимся в виде печатных или электронных учебных пособий,
  что позволяет либо полностью отказаться от установочных сессий с приездом в ВУЗ, либо
  значительно сократить их число и длительность.
\end{itemize}

К числу недостатков дистанционной системы обучения сегодня можно конечно отнести:
\begin{itemize}[wide,topsep=0pt]
  \itemsep0em
  \item сужение потенциальной аудитории учащихся, которое объясняется отсутствием технической
  возможности включения в учебный процесс (компьютер, Интернет-связь);
  \item обязательность компьютерной подготовки как необходимого условия вхождения в систему
  дистанционного образования;
  \item неадаптированность учебно-методических комплексов к
  учебным курсам дистанционного образования (в частности электронных учебных пособий).
  \item недостаточная разработанность систем администрирования учебного процесса и, как результат,
  снижение качества дистанционного образования в сравнении с очным обучением.
\end{itemize}

Очень серьезной проблемой дистанционного обучения является переосмысление использования многих
проверенных педагогических приемов для лучшего запоминания и усвоения материала,
таких, как: метод опорных точек, метод сознательных ошибок, метод выбора лучшего
решения и т. д. Применение самых различных педагогических методов становится в большей
степени зависимым от технических средств и способов организации контакта с обучаемыми.
Однако необходимо отметить при любой технологии взаимодействия преподавателю приходится
учиться более сжато и четко излагать материал или отвечать на вопросы. И в данной ситуации
становится концептуальным постоянное и непрерывное самосовершенствование как преподавателя,
так и обучающегося.

\subsection{Существующие платформы для онлайн обучения}

Рынок отвечает на растущий спрос и во всём мире появляются крупные ресурсы, на платной и
свободной основе. Более того, университеты Лиги Плюща, такие как Гарвард и Йель предлагают
специализированные курсы по гуманитарным и точным наукам на самых разных платформах. Ориентируясь
на существующие тренды, рассмотрим некоторые популярные платформы для онлайн обучения.

\subsubsection{Coursera}

Проект в сфере массового онлайн-образования был запущен в 2012 году. Coursera основана
профессорами информатики Стэнфордского университета Эндрю Ыном и Дафной Коллер.
На базе сотрудничества с широким спектром университетов, онлайн-платформа генерирует
образовательные материалы, сформированные в систему курсов, которые проходят студенты.
На сайте Coursera проводятся тесты и экзамены. Также имеется официальное мобильное
приложение для iPhone и Android.\cite{the-steppe}

В проекте представлены полноценные курсы по физике, инженерным дисциплинам, гуманитарным наукам,
искусству, медицине, биологии, математике, информатике, экономике и бизнесу.
Продолжительность курсов варьируется от шести до десяти недель, куда входят видео-лекции,
презентации, домашние задания и тексты. Предусмотрены чаты с сокурсниками, где можно обсуждать
идеи и помогать с решением задач. Курсы можно пройти как на английском, так и на русском языке,
зачастую с субтитрами. Coursera сотрудничает с университетами Стэнфорда, Принстона, Мичигана,
Пенсильвании и многими другими. По окончанию обучения студенты получают соответствующие
сертификаты и дипломы. Это может быть степень бакалавра или магистра.

По последним данным, Coursera насчитывает более 25 миллионов пользователей,
149 университетов-партнёров, около 2000 тысяч различных курсов по 180 специализациям.
Большая часть курсов основана на платной основе, но есть и набор семинаров, лекций, доступных
бесплатно.

\subsubsection{Khan Academy}

Некоммерческая образовательная академия, созданная в 2006 году выпускником MIT и Гарварда
Салманом Ханом, стремится предоставлять высококачественное образование каждому. На сайте
академии можно найти более 4200 бесплатных мини-лекций на различные темы. Пользователи
могут изучить особенности квантовой физики или погрузиться в исторические сводки глобальных
конфликтов в период 1907-1960 годов. Платформа станет настоящей находкой для всех тех,
кто интересуется культурой, искусством и литературой. Обзорные видео расскажут о творчестве
таких легендарных личностей как Пикассо, Маттисс и Дали. Циклы материалов можно прослушать
на русском языке. Кроме того, в курсы включены статьи от известных авторов в своих областях.

Проект спонсируется компанией Google и Фондом Билла и Мелинды Гейтс. Кроме этого,
Khan Academy принимает пожертвования от пользователей, чтобы поддерживать некоммерческую
модель и генерировать актуальный контент.

\subsubsection{Udacity}

Udacity — частная образовательная организация, основанная Себастьяном Труном,
Дэвидом Ставенсом (David Stavens) и Майклом Сокольски (Mike Sokolsky),
с целью демократизации образования. Компания возникла в результате расширения
программы по информатике Стэнфордского университета.
Дистанционные курсы доступны бесплатно по Интернету, прослушать их может любой желающий.
Вначале предлагалось шесть курсов. На 1 октября 2012 года Udacity предлагает 14 курсов.
Число студентов составляет десятки тысяч человек.\cite{udacity}

В 2012 году Себастьян Трун был отмечен газетой The Guardian как человек, внёсший
существенный вклад в развитие открытого Интернета.

Видео лекции на английском языке с субтитрами в сочетании со встроенными тестами
и последующими домашними работами, основанные на модели «учиться на практике».
Каждая лекция включает в себя встроенный тест, чтобы помочь студентам понять предлагаемые
концепции и идеи.

К искусственному интеллекту, робототехнике и физике добавились курсы по ведению блогов и
созданию стартапов, но не всё идет гладко. Многие идеи провалились, как, например,
курс по дискретной математике — уровень оказался слишком низким по сравнению с остальными.
С другой стороны, на форумах до сих пор много недовольных: курсы, по их мнению, находятся
за гранью возможностей и искусственно усложнены. Команда проекта принимает это к
сведению — проще, конечно, не будет, но способы изложения материала и тесты постоянно
пересматриваются, чтобы как можно больше студентов смогли освоить предмет, несмотря
на сложность.

Студенты могут зарегистрироваться на один или несколько классов до даты сдачи первого
домашнего задания. По окончании курса студенты бесплатно получают сертификат об окончании,
подписанный преподавателями.

\subsubsection{Codecademy}

Ресурс был создан в августе 2011 года Заком Симсом и Райяном Бубински.
Чтобы сфокусироваться на проекте, оба бросили Колумбийский университет.
Codecademy является интерактивной онлайн-платформой по обучению 12 языкам
программирования: Python, PHP, JavaScript, Ruby, Java и др. Курсы проходят
на английском языке. Ресурс высоко ценится на мировом рынке веб-индустрии.
New York Times и TechCrunch положительно отзываются об Codecademy.
Пользователи после регистрации получают свой уникальный профиль. В онлайн-обучение
имплементирована система поощрительных достижений за выполнение различных упражнений,
которые могут видеть другие пользователи. Также на сайте есть форум, где новички
могут задать свои вопросы и получить обратную связь от экспертов в нишевых областях.
Для некоторых курсов создана «песочница», где пользователи могут тестировать свои
программные коды.

\subsubsection{Универсариум}

Крупнейшая российская платформа онлайн-обучения была запущена в 2013 году.
Выбор курсов довольно большой: от фундаментальных наук, таких как физика,
химия, математика, до нишевых курсов по робототехнике и авиамоделированию.
Универсариум предлагает полноценные бесплатные курсы, выполненные по образовательным
стандартам электронного обучения, которые включают видео-лекции, самостоятельные
задания, тесты, групповую работу и итоговую аттестацию. Курсы составлены из модулей,
где каждый модуль длится одну неделю. Проект реализуется при поддержке РИА Наука и Агентства
стратегических инициатив.

На платформе Универсариума представлены курсы 46 университетов, включая МГУ, НИУ ВШЭ, МФТИ
и компании Mail.ru. После окончания курса выдается электронный сертификат.
Все обучение на платформе бесплатное. У проекта есть мобильное приложение.