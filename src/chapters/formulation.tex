\section*{ГЛАВА 1}
\setcounter{section}{1}\setcounter{subsection}{0}
\addcontentsline{toc}{section}{ГЛАВА 1}

\subsection{Постановка задачи}

Процесс дистанционного обучения (учебно-воспитательный процесс) характеризуется, в первую очередь
тем, что он интерактивен в своей организации, т. е. во взаимодействии учителя и ученика, а
также учащихся между собой, имеет конкретную предметную область познания. Следовательно,
когда мы говорим о процессе дистанционного обучения, мы предполагаем наличие в этом процессе
преподавателя и учащихся, их общение, общение учащихся между собой. Можно сделать вывод о
необходимости создания единого информационно-образовательного пространства, включающего в
себя всевозможные электронные источники информации (в том числе, сетевые): виртуальные
библиотеки, разнообразные базы данных, видеолекции, электронные учебные пособия и пр.

Т.к тема диссертации достаточно обширна и в зависимости от размера образовательной платформы
и направления её деятельности, система может включать десятки дополнительных модулей,
то в данной работе были сформулированы следующее задачи:
\begin{itemize}[wide,topsep=0pt]
  \itemsep0em
  \item сделать обзор предметной области и провести анализ существующих программных решений в сфере дистанционного образования;
  \item провести выбор основных средств и методов разработки;
  \item разработать простой, интуитивно понятный каркас системы, с минимальным количеством модулей,
    необходимых для полноценной работы образовательной платформы. А так же решить основные
    проблемы, которые препятствовали ли бы быстрому расширению системы.
\end{itemize}

Далее рассмотрим более подробно каждый из этих пунктов.


\subsection{Обзор предметной области}

В данный момент существует достаточно много различных решений для дистанционного обучения.
Фактически это класс систем для просмотра видеолекций, прохождения онлайн-курсов,
чтения электронной книг.

Перед началом работы необходимо провести анализ существующих платформ,
выделить их сильные и слабые стороны.


\subsection{Выбор основных средств и методов разработки}

Одна из сложных задач, которая возникает при внедрении программного средства,
это как передать продукт клиенту. Предположим у вас есть проект, который вы закончили
и теперь его необходимо передать пользователю. Вы готовите много разных файликов,
скриптов и пишите инструкцию по установке. А потом тратите уйму времени на решения
проблем клиента вроде: «у меня ничего не работает», «ваш скрипт упал на середине —
что теперь делать», «я перепутал порядок шагов в инструкции и теперь не могу идти дальше»
и т. п. Всё усугубляется если продукт тиражируемый и вместо одного клиента у вас их десятки или
сотни. И становится еще сложнее, если вспомнить о необходимости установки новых версий продукта.

Другой сложно задачей при разработке видеоплатформы является обработка, хранения и доставка
видео. Сервера должны расположены по всему миру таким образом, чтобы время ответа посетителям
из разных стран было минимальным. Пользователи не должны испытывать никаких задержек при
просмотре видеолекций.

Существует несколько различных подходов к реализации данных задач.
Выбор правильного метода позволит съэкономить бюджет при развертывании инфраструктуры
приложения, а так же уложиться в сроки, выделенные для процесса внедрения образовательной
платформы в учебное заведение.

Таким образом, при изучении данной темы необходимо провести тщательный
анализ существующих технологий по обработке, хранению и доставке видео, а также методов для
быстрого развертывания необходимой инфраструктуры в рамках конкретного учебного заведения.

\subsection{Анализ необходимых модулей и дополнительных требований}

За основу была выбрана видеоплатформа. Минимальный набор модулей, необходимых
для просмотра видеолекций, управления пользователями и видеофайлами включает:

\begin{itemize}[wide,topsep=0pt]
  \itemsep0em
  \item регистрация и авторизация пользователей в роли студента или лектора;
  \item разграничение прав по всем возможным объектам
  \item создание тематических каналов;
  \item загрузка видео на выбранном канале;
  \item разбивка видео на тематические блоки и секции;
  \item возможность написания комментария для определенного таймфрейма видео;
  \item навигация по блокам и секциям при просмотре видео;
  \item просмотр комментариев для текущего блока/секции видео;
  \item управление пользователями, видео материалами, комментариями через панель администрирования платформы.
\end{itemize}

Так же, т.к очевидно, что система будут расширяться, в зависимости от нужд учреждения,
необходимо реализовать механизм, который позволил бы с минимальными усилиями разрабатывать
новые модули для приложения, достаточно просто описывать их и интегрировать
в уже существующие процессы.

Приложение должно позволять работать сотрудникам используя разные операционные системы,
а так же должны быть продуманы возможности будущих интеграций.
