\section{ПОСТАНОВКА ЗАДАЧИ}
\addcontentsline{toc}{section}{Постановка задачи}

В данном разделе будет произведён обзор предметной области задачи, решаемой в рамках диссертационного проекта.

\subsection{Анализ существующих образовательных платформ}

Процесс дистанционного обучения (учебно-воспитательный процесс) характеризуется, в первую очередь
тем, что он интерактивен в своей организации, т. е. во взаимодействии учителя и ученика, а
также учащихся между собой, имеет конкретную предметную область познания. Следовательно,
когда мы говорим о процессе дистанционного обучения, мы предполагаем наличие в этом процессе
преподавателя и учащихся, их общение, общение учащихся между собой. Можно сделать вывод о
необходимости создания единого информационно-образовательного пространства, включающего в
себя всевозможные электронные источники информации (в том числе, сетевые): виртуальные
библиотеки, разнообразные базы данных, консультационные службы, электронные учебные пособия и пр.
