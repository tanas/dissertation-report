\section*{ОБЩАЯ ХАРАКТЕРИСТИКА РАБОТЫ}
\addcontentsline{toc}{section}{Общая характеристика работы}

\subsection*{Цель и задачи исследования}

\textit{Целью} диссертационной работы является разработка платформы для просмотра видеоматериалов,
общения преподавателей и студентов в сфере дистанционного образования.

Для достижения поставленной цели необходимо решить следующие задачи:
\begin{itemize}
  \item изучить существующее платформы дистанционного обучения;
  \item выявить актуальные проблемы при разработке образовательной платформы и предложить их решение;
  \item разработать каркас образовательной платформы для просмотра видеоматериалов.
\end{itemize}

\textit{Объектом} исследования являются обучающие платформы в сфере дистанционного образования.


\textit{Предметом} исследования являются методы храниения, обработки и просмотра видеоматериалов.

Основной \textit{гипотезой}, положенной в основу диссертационной работы, является возможность разработки
модулей для динамического расширения объектов, а также программного обеспечения для удобного
управления образовательной платформой, с использованием современных подходов в разработке
и проектировании программного обеспечения.

\subsection*{Связь работы с приоритетными направлениями научных исследований и запросами реального сектора
экономики}

Работа выполнялась в соответствии научно-техническими заданиями и планами работ
Частного предприятия «Смарт АйТи». Программное средство и решения, разработанные в процессе работы,
были внедрены на производство. Результаты внедрения были отражены в акте о внедрении.

\subsection*{Личный вклад соискателя}

Результаты, приведенные в диссертации, получены  соискателем лично.
Вклад научного руководителя В. Н. Теслюк, заключается в формулировке целей и задач исследования.

\subsection*{Апробация результатов диссертации}

Составные части диссертационной работы докладывались и обсуждались
на 52й Студенческой международной научно-практической конференции
«Молодежный научный форум: Технические и математические науки» (Москва, Россия, 2018).

\subsection*{Опубликованность результатов диссертации}

По теме диссертации опубликовано 1 печатная работа в сборниках трудов и материалов
международной конференций.


\subsection*{Структура и объем диссертации}

Диссертация состоит из введения, общей характеристики работы, четырех глав, заключения,
списка использованных источников, списка публикаций автора, приложения и акта о внедрение результатов
работы в производстве. В первой главе описана постановка задачи, произведен анализ необходимых
модулей и дополнительных требований. Во второй главе произведен анализ предметной области,
описаны существующие решения, выделены их сильные и слабые стороны.
Третья глава посвящена способам внедрения ERP систем, описаны различные подходы к внедрению
систем управления, предложен собственный метод внедрения.
В четвертой главе рассмотрены вопросы разработки архитектуры ПО и реализации алгоритма
динамического расширения объектов.
В рамках решения практической задачи, описанные алгоритмы реализованы в виде отдельного
программного средства.

Общий объем работы составляет 60 страниц, из которых основного текста – 52 страницы,
8 рисунков на 8 страницах, список использованных источников из 23 наименований на 3 страницах.
