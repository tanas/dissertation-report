\section*{ЗАКЛЮЧЕНИЕ}
\addcontentsline{toc}{section}{ЗАКЛЮЧЕНИЕ}

Результатом данной магистерской диссертации является образовательная платформа со следующими
возможностям:
\begin{itemize}[wide,topsep=0pt]
  \itemsep0em
  \item авторизация пользователей в роли студента, лектора или администратора;
  \item разграничение прав на просмотр, аннотирование, редактирование видео;
  \item создание тематических каналов и загрузка видеолекций;
  \item разбивка видео на тематические блоки и секции;
  \item навигация по блокам и секциям при просмотре видео;
  \item возможность комментирования видео с привязкой к определенному таймфрейму;
  \item просмотр комментариев для текущего блока/секции видео;
  \item управление пользователями, видеоматериалами, комментариями через панель администрирования.
\end{itemize}

Интерфейс приложения не перегружен сложными элементами, что позволяет начать работать
с приложением без дополнительной подготовки.
В то же время функциональные возможности, реализованные по умолчанию, покрывают основные
задачи образовательной платформы и при необходимости могут быть расширены за счет модульной
структуры.

В ходе работы были исследованы современные образовательный платформы,
выявлены их слабые и сильные стороны. Был проведен анализ существующих способов обработки
и доставки видео и предложен наиболее подходящий метод.

В ходе работы были исследованы методы обеспечения работоспособности приложения в различных
средах. Была реализована контейнеризация приложения. Это позволило снизить время-затраты
по развертыванию инфраструктыры, расширению системы и интеграции со сторонними
продуктами.

Доработанное программное средство было внедрено на производстве и получило положительные
оценки пользователей. Вместе с тем, за время его использования, были собраны
пожелания и замечания, которые будут учтены в следующих версиях.
Разработанный каркас системы был представлен заинтересованным командам и в настоящее
время используется для разработки новых приложений.

В итоге получилось раскрыть тему магистерской диссертации и на практике реализовать
описанные идеи. Собранный теоретический материал был представлен на международной
конференции в г. Москва.

В дальнейшем планируется развивать и усовершенствовать приложения, разработаывать новые
и дорабатывать текущие модули.


