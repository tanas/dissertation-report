\section*{ЗАКЛЮЧЕНИЕ}
\addcontentsline{toc}{section}{ЗАКЛЮЧЕНИЕ}

Таким образом в ходе работы была разработана образовательная платформа со следующим функционалом:
\begin{itemize}[wide,topsep=0pt]
  \itemsep0em
  \item регистрация и авторизация пользователя в роли студента или лектора;
  \item создание тематических каналов;
  \item загрузка видео на выбранном канале;
  \item разбивка видео на тематические блоки и секции;
  \item возможность написания комментария для определенного таймфрейма видео;
  \item навигация по блокам и секциям при просмотре видео;
  \item просмотр комментариев для текущего блока/секции видео;
  \item управление пользователями, видео материалами, комментариями через панель администрирования платформы.
\end{itemize}

Интерфейс приложения не перегружен сложными элементами, что позволяет начать работать с приложением
без дополнительной подготовки. В то же время функциональные возможности, реализованные
по умолчанию, покрывают основные задачи образовательной платформы и при необходимости могут быть
расширены, за счет модульной структуры.