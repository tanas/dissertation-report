\section*{ВВЕДЕНИЕ}
\addcontentsline{toc}{section}{Введение}

В современном мире профессиональные знания быстро устаревают,
необходимо их постоянное совершенствование.
Дистанционная форма обучения дает сегодня возможность создания систем массового непрерывного
самообучения, всеобщего обмена информацией, коммуникации студентов и преподавателей,
независимо от наличия временных и пространственных поясов.
Прослушать лекции знаменитых профессоров, получить повышение квалификации
в ведущих университетах мира, воспользоваться обширными электронными библиотеками
или поучаствовать в вебинаре – все это становится доступным благодаря дистанционному обучению.

В настоящее время существует большое количество образовательных проектов, сайтов,
где можно послушать лекции именитых профессоров.
Одна из задач, которая ставилась в данной работе – разработать простую,
интуитивно понятную систему, в которой была бы возможность не только просмотреть
видео лекции, но и пообщаться, задать вопросы преподавателю, оставить комментарий.
Отличительной особенностью является то, что и сам лектор и любой студент могут оставлять
комментарии прямо при просмотре лекции.
Комментарий сохранится с привязкой к таймфрейму и будет отображен при просмотре
соответствующего фрагмента видео.
Это может быть как пояснение к сложному моменту лекции, сделанное лектором,
так и заданный студентом вопрос, с ответом на который сможет помочь любой желающий.
