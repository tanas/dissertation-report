\section*{ВВЕДЕНИЕ}
\addcontentsline{toc}{section}{ВВЕДЕНИЕ}

В современном мире профессиональные знания быстро устаревают,
необходимо их постоянное совершенствование.
Дистанционная форма обучения дает сегодня возможность создания систем массового непрерывного
самообучения, всеобщего обмена информацией,
независимо от наличия временных и пространственных поясов.
Прослушать лекции знаменитых профессоров, получить повышение квалификации
в ведущих университетах мира, воспользоваться обширными электронными библиотеками
или поучаствовать в вебинаре – все это становится доступным благодаря дистанционному обучению.

Дистанционное обучение — взаимодействие учителя и учащихся между собой на расстоянии,
отражающее все присущие учебному процессу компоненты (цели, содержание, методы,
организационные формы, средства обучения) и реализуемое специфичными средствами
Интернет-технологий или другими средствами, предусматривающими интерактивность.

В настоящее время существует большое количество образовательных проектов c сфере дистанционного
обучения, где можно прослушать лекции именитых профессоров, пройти интерактивные курсы обучения.
Как правило это глобальные платформы, направленные на привлечение аудитории со всего мира.
Одна из задач, которая ставилась в данной работе – разработать простую,
интуитивно понятную систему, которая может быть внедрена в рамках отдельного университета.


Платформа должна способствовать коммуникации студентов и преподавателей.
В ней должна быть возможность подписки на видеокурсы и просмотра лекций.
Отличительной особенностью будет является то, что лектор сможет аннотировать загруженную лекцию.
Аннотации разбивают видео на логические фрагменты: блоки и секции.
Их задача пояснять сложные моменты лекции, давать более подробное описание по текущей теме.
Блоки и секции добавляются с привязкой к таймфрейму и отображаются при просмотре соответствующего
фрагмента видео.
Студент, в свою очередь, будет иметь возможность задать вопрос, относящиеся к определенному
блоку видео, помочь с ответом на который сможет любой желающий.

Система должна быть доступной и не требовать больших трудозатрат для её внедрения и поддержки.
В тоже время использование современных технологий при разработке и модульная структура
системы должны позволять с минимальными затратами расширять данную систему под нужды
конкретного университета.

Связь исследуемого предмета с современной системой образования дает возможность
провести проверку полученных результатов и внедрить разработанное приложение в
отдельный университет.
