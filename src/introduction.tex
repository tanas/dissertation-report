\section*{ВВЕДЕНИЕ}
\addcontentsline{toc}{section}{Введение}

В последние время, благодаря интенсивному развитию информационных технологий, в интернете можно купить практически все. Интернет-магазины уверенно держат свои позиции, ведь удобство такого шопинга оценили уже многие. Если ранее, чтобы приобрести билеты, необходимо было потратить достаточно много времени, сейчас данная задача значительно упрощена. Очереди и пробки на дорогах сменяются уютной атмосферой дома или офиса. Заказать необходимый билет можно в любое время суток и в любом месте.

Если вы задумываетесь о покупке билетов на мероприятия, будь то концерты, спектакли, сеансы в кинотеатре, семинары или выставки, вы наверняка и сами понимаете, что предпосылок отказаться от классической покупки, в пользу онлайн более чем достаточно. Проникновение Интернета в нашу жизнь становиться все более заметным. Объемы онлайн торговли растут с каждым днем, и мало кто захочет быть в стороне от этого процесса.

Преимущества, конечно же, очевидны, но все же стоит их перечислить:
\begin{itemize}
  \item Экономия времени. Для совершения онлайн покупки билетов через сайт продавца, нужно потратить 3-4 минуты. Что бы купить билет в кассе, надо потратить минимум 30 минут, обычно гораздо больше, особенно в крупных городах.
  \item Экономия денег. При покупке или бронировании билета традиционным способом, надо потратить деньги на транспорт или услуги курьера. Эта сумма, как правило, превышает размер комиссии сервисов.
  \item Возможность выбора мест. Пользователь сам выбирает понравившиеся места. 
  \item Возможность купить билеты заранее, и спланировать свое время рационально. 
\end{itemize}

Однако, всё ли так просто? Существует большое количество сервисов, которые предлагают довольно узкий спектр услуг, например, предложение купить билет в определённый кинотеатр или театр. Это, конечно, тоже весьма неплохо, но ориентироваться в огромном количестве онлайн-сервисов весьма непросто. Хотелось бы видеть систему, которая смогла бы предложить покупку билета на большой спектр мероприятий, будь то спектакль, концерт или выставка.

Но, опять же, с технической стороны к такой системе предъявляются серьёзные требования. Необходимо аггрегировать большое количество разноплановых подсистем, которые взаимодействуют с различными заведениями: кинотеатрами, театрами, концертными залами и др. 

Целью данного дипломного проекта являются описание и разработка программного обеспечения по управлению централизованными продажами через систему Bycard.

Была поставлена задача разработать такую систему, которая бы могла не только подключать заведения, но и позволила бы разрабатывать приложения для продажи билетов, для показа актуальной афиши с минимальными затратами. Это может быть любой клиент: сайт, мобильное приложение, информационный ресурс.

В данном дипломном проекте рассматривается процесс проектирования и реализации данной системы. Большое внимание уделяется контролю качества, созданию автоматической системы, которая контролирует корректность работы всех внешних и внутренних модулей, интегрированных в проект.

\newpage
